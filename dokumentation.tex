\documentclass[12 pt]{article}
\usepackage[utf8]{inputenc}
\usepackage{geometry}



\title{\small 62532 VERSIONSSTYRING OG TEST METODER (E22) \vskip 5pt
	\huge{\textbf{CDIO 1} \vskip 5pt
		\small DTU - Danmarks Tekniske Universitet}}

\author{Negar Ostadhassan Salmani - s224283\\ 
	Mélissa Woo - s224311\\
	Amira Omar - s205821 \\
	Besma Faris Al-Jwadi - s224325 \\
	David Rivera - s215461}

\date{September 2022}



\newgeometry{top=2.5cm, bottom=2.5cm, left=2.5cm, right=2.5cm}
\begin{document}
	
	\maketitle
	\newpage
	\begin{enumerate}
		\item 	Spørgsmål til projekt leder:
		\begin{enumerate}
			\item \textbf{Hvordan definerer kunden et 'almindeligt menneske'? (fastlås hvad et ‘almindeligt menneske’ er, et definition man kan hænge noget op på)} \vskip 1pt
			Den 19.09.22: Skal spørge kunden.
			
			\item \textbf{Hvad betyder det, at spillet 'virker korrekt'?} \vskip 1pt
			De 2 terninger skal følge 2 terningers normalfordeling indenfor en procentavigelse på 1 procent. 
			
			\item \textbf{Er vinderen den, der opnår 40 points først, eller er der flere runder?} \vskip 1pt
			Den 19.09.22: Skal spørge kunden.
			
			\item \textbf{Skal man have en specifik alder for at kunne spille?} \vskip 1pt
			Den 19.09.22: Skal spørge kunden.

			\item \textbf{Hvad sker der, når man vinder? Penge? ECTS points?} \vskip 1pt
			Den 19.09.22: Skal spørge kunden.
			
			\item \textbf{Hvem spiller mod hvem?} \vskip 1pt
			Det er 2 menneskespillere, der spiller mod hinanden (ikke AI)
		\end{enumerate}
	\end{enumerate}
	
	
\end{document}